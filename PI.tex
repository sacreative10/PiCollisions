% This file was converted to LaTeX by Writer2LaTeX ver. 1.9.9
% see http://writer2latex.sourceforge.net for more info
\documentclass{article}
\usepackage{calc,amsmath,amssymb,amsfonts}
\usepackage[LGR,T1]{fontenc}
\usepackage[greek,english]{babel}
\usepackage[style=numeric,backend=biber]{biblatex}
\usepackage{array,supertabular,hhline}
\usepackage[pdftex]{graphicx}
\setlength\tabcolsep{1mm}
\renewcommand\arraystretch{1.3}
\date{2023-11-26}
\begin{document}

\bigskip


\bigskip

\title{PI from collisions}
\maketitle


\bigskip

\section{Motivation}
This project was motivated by the idea of creating a simulation that approximates PI (for certain definitions of
reasonable), and this specfic project arises from a interesting problem postulated by a maths youtuber named 3b1b.


\bigskip

The premise behind this simulation is as follows. Imagine two blocks, A and B, where the ratio between their masses,
denoted mA and mB, be 100N (mB / mA). Push the latter block towards the other, assuming a frictionless elastic world,
where of A$\text{\textgreek{’}}$s left hand side there is a wall perpendicular to the ground. From this, the number of
collisions within this system must be N digits of $\pi $.


\bigskip

I won$\text{\textgreek{’}}$t delve deep into why this occurs, but rather focus on how to make it occur (so to speak). In
order to learn more, please read the original paper, and said video.


\bigskip

I will be using Python, as well the excellent Manim community edition. I use the former due to its simplicity, and the
latter because it might be the only decent “graphics” library available in Python. Fun fact: The original author of the
Manim library was also the person who made the video!


\bigskip

Ideally, I would like this project to be simple, yet elegant. I do not wish for this program to be a giant monad of
sorts. What I mean is that, I don$\text{\textgreek{’}}$t what this program to be a “machine” with many distinct moving
parts. Yes, there will obviously be seperate classes, functions, but I want them to be self-contained, rather than a
“spagetti” of sorts.

 \includegraphics[width=12.7cm,height=9.208cm]{PI-img001.png} 


\bigskip

\section{Module 1: Rendering An Screen}

\bigskip


\bigskip

\subsection{Test plan}
Now ideally we want the window to remain open until closed, as well as for it to exit when escape it pressed. For a
basic window, that is all we want.

\begin{center}
\tablefirsthead{}
\tablehead{}
\tabletail{}
\tablelasttail{}
\begin{supertabular}{m{1.091cm}m{5.307cm}m{4.598cm}m{2.504cm}m{2.501cm}}
TC(n) &
Test data &
Expected Output &
Evidence &
Passed?\\
1 &
Start window &
Should remain open &
~
 &
~

~
\\
2 &
Start window, press escape &
Should close &
~
 &
~
\\
\end{supertabular}
\end{center}

\bigskip


\bigskip


\bigskip

\subsection{Development}

\bigskip

We would like a window (preferably an openGL instance) of a window. Fortunatley, Manim has a feature that allows us to
render to a window (rather than to a video, as default):



\begin{figure}
\includegraphics[width=17cm,height=3.799cm]{PI-img002.png}\end{figure}
Where we create a sample scene class, and “play” anything that needs to be moved, animating a video, that is then played
on top of the opengl window.


\bigskip

However all test cases for this module fail. The reason being manim is not event-driven, rather (as explained), it
simply plays a video*. Therefore it is not possible for us to do anything related to events and capturing said events.
Maybe we could run the video on a loop constantly (like in a seperate pygame window), however, as discussed before,
that would mean breaking the simplicity rule, as this would be a crummy solution to a dead simple problem. Therefore, I
elected to use PyGlet, a Pythonic graphics library.


\bigskip

Pyglet is built in a similar vein to PyGame, however, I chose the former over the latter, due to the
former$\text{\textgreek{’}}$s better control for deltatime, something which will prove useful in later modules.


\bigskip

Since Pyglet (along with most of python$\text{\textgreek{’}}$s libraries) are purely imperative, I will create a window
class, in order for ease of use, as well as clean code. Also, Pyglet$\text{\textgreek{’}}$s documentation recommend
subclassing the window.

\begin{figure}
\includegraphics[width=12.199cm,height=12.1cm]{PI-img003.jpg}\end{figure}

\bigskip


\bigskip


\bigskip


\bigskip


\bigskip


\bigskip


\bigskip


\bigskip


\bigskip


\bigskip

Firstly, we have the initializer. As you can see, I have added the ability for the client programmer (me), to expand the
input handling, as well as expanding the drawing, since I feel like that is a good way to abstract Pyglet out of the
equation in order to progress to the more math-heavy code later on.



\begin{figure}
\includegraphics[width=13.653cm,height=2.752cm]{PI-img004.png}\end{figure}

\bigskip



\begin{figure}
\includegraphics[width=17cm,height=6.101cm]{PI-img005.png}\end{figure}
And this is how we (for example), allow the client to expand their keyboard handling, and of course we have included the
criteria for the window to close when escape is pressed as part of our test plan.



\begin{figure}
\includegraphics[width=17cm,height=3.201cm]{PI-img006.png}\end{figure}
This is how we create a window. Note that we also have checks to see if Pyglet failed with its initialisation.


\bigskip

\subsection{Module 1 Testing}
For the test code, I have written only 3 lines, one to initialise the class, one to initialise the window, and one to
start pyglet proper.



\begin{figure}
\includegraphics[width=8.864cm,height=1.693cm]{PI-img007.png}\end{figure}

\bigskip


\bigskip


\bigskip

\begin{center}
\tablefirsthead{}
\tablehead{}
\tabletail{}
\tablelasttail{}
\begin{supertabular}{m{1.0879999cm}m{5.303cm}m{4.5940003cm}m{3.1cm}m{1.915cm}}
TC(n) &
Test data &
Expected Output &
Evidence &
Passed?\\
1 &
Start window &
Should remain open, until closed &
~
 &
~

~
\\
2 &
Start window, press escape &
Should close &
~
 &
~
\\
\end{supertabular}
\end{center}

\bigskip


\bigskip

Note: evidence needs adding


\bigskip


\bigskip

\section{Module 2: Conserving Momentum}
Now, we enter the fun part of the project, by conserving momentum. We need to do so, in order to satisfy the first
premise of our problem.

\ 

\begin{figure}
\includegraphics[width=11.642cm,height=3.519cm]{PI-img008.png}\end{figure}

\bigskip


\bigskip


\bigskip


\bigskip

These equations are used in an elastic environment, where both kinetic energy and momentum are equal before and after
any collision. Vi describes the final velocity (after a collision) of the ith particle, mi the same for mass, and ui
for initial velocity.

In foresight, it might be prudent for us to represent the current velocities and masses of our particles in some sort of
structure, in order to make legibility and further expansion easier.



\begin{figure}
\includegraphics[width=15.505cm,height=15.452cm]{PI-img009.png}\end{figure}


\begin{figure}
\includegraphics[width=17cm,height=9.197cm]{PI-img010.png}\end{figure}

\bigskip

Therefore, our test plan is as follows:

\begin{flushleft}
\tablefirsthead{\ TC(n)  &
\ Test name \ \ \ \ \ \ \ \ \ \ \ \ \ \ \ \ \ \ \ \ \ \ \ \ \ \ \ \ \ \ \ \  &
\ Test data \ \ \ \ \ \ \ \ \ \ \ \ \ \ \ \ \ \ \ \ \ \ \ \ \ \ \ \ \ \ \ \ \ \ \ \ \ \ \ \  &
Expected Output\\}
\tablehead{\ TC(n)  &
\ Test name \ \ \ \ \ \ \ \ \ \ \ \ \ \ \ \ \ \ \ \ \ \ \ \ \ \ \ \ \ \ \ \  &
\ Test data \ \ \ \ \ \ \ \ \ \ \ \ \ \ \ \ \ \ \ \ \ \ \ \ \ \ \ \ \ \ \ \ \ \ \ \ \ \ \ \  &
Expected Output\\}
\tabletail{}
\tablelasttail{}
\begin{supertabular}{m{0.83cm}m{7.039cm}m{4.4430003cm}m{0.8880001cm}}
\ 3 \ \ \  &
\ test\_collision\_with\_equal\_mass \ \ \ \ \ \ \ \ \ \ \  &
\ Object(1, 2), Object(1, -3) \ \ \ \ \ \ \ \ \ \ \ \ \ \ \ \ \ \ \ \ \ \  &
(-3, 2)\\
\ 4 \ \ \  &
\ test\_collision\_with\_different\_mass \ \ \ \ \ \ \  &
\ Object(2, 4), Object(1, -3) \ \ \ \ \ \ \ \ \ \ \ \ \ \ \ \ \ \ \ \ \ \  &
(0.5, 3.5)\\
\ 5 \ \ \ \  &
\ test\_collision\_with\_one\_stationary\_object  &
\ Object(2, 0), Object(1, -3) \ \ \ \ \ \ \ \ \ \ \ \ \ \ \ \ \ \ \ \ \ \  &
(-3, 0)\\
\ 6 \ \ \ \  &
\ test\_collision\_with\_opposite\_directions \ \  &
\ Object(1, 2), Object(1, 3) \ \ \ \ \ \ \ \ \ \ \ \ \ \ \ \ \ \ \ \ \ \ \  &
(3, 2)\\
\ 7 \ \ \ \  &
\ test\_collision\_with\_same\_directions \ \ \ \ \ \  &
\ Object(1, 2), Object(1, 2) \ \ \ \ \ \ \ \ \ \ \ \ \ \ \ \ \ \ \ \ \ \ \  &
(2, 2)\\
\ 8 \ \ \ \  &
\ test\_collision\_with\_large\_mass\_difference  &
\ Object(100, 5), Object(1, -3) \ \ \ \ \ \ \ \ \ \ \ \ \ \ \ \ \ \ \ \  &
(-3, 5)\\
\ 9 \ \ \ \  &
\ test\_collision\_with\_zero\_velocity \ \ \ \ \ \ \ \  &
\ Object(2, 0), Object(1, 0) \ \ \ \ \ \ \ \ \ \ \ \ \ \ \ \ \ \ \ \ \ \ \  &
(0, 0)\\
\end{supertabular}
\end{flushleft}
\subsection{Module 2: Development}

\bigskip

I decided to program the Object as a data structure rather than a class, since I felt that since this object is really
here to store data, there isn$\text{\textgreek{’}}$t an advantage of creating a class with a constructor, as this is
only here to store data in an organised container. If this object had more than getters and setters as its method, a
class would have been beneficial. Using a data structure not only is faster, but is also more legible.

\ 

\begin{figure}
\includegraphics[width=6.509cm,height=2.831cm]{PI-img011.png}\end{figure}
Note that we only use one dimensional vectors, as our boxes will only exist upon a line, and won$\text{\textgreek{’}}$t
be flying. If we needed the y component, we could use sin(velocity) to yield our y component.


\bigskip


\bigskip

\ 

\begin{figure}
\includegraphics[width=17cm,height=5.274cm]{PI-img012.png}\end{figure}
Then we have the solver following the equation above. Note that we \ use type hinting of a tuple in order to return both
velocities.


\bigskip

\subsection{Module 2: Testing}
I used unittests for this module, as firstly, they provide comprehensive tools for testing in a orderly manner.
Secondly, we are testing a (mathematical) function, which requires more thorough testing.

Writing the test was rather difficult with python$\text{\textgreek{’}}$s module structure, therefore I ended up with
this structure (ignore pycache):

where we need \_\_init\_\_.py in each subfolder to indicate its module status. Very annoying initially, and will remain
annoying.

\begin{figure}
\includegraphics[width=17cm,height=7.756cm]{PI-img013.png}\end{figure}

\bigskip

\begin{flushleft}
\tablefirsthead{\ TC(n)  &
\ Test name \ \ \ \ \ \ \ \ \ \ \ \ \ \ \ \ \ \ \ \ \ \ \ \ \ \ \ \ \ \ \ \  &
\ Test data \ \ \ \ \ \ \ \ \ \ \ \ \ \ \ \ \ \ \ \ \ \ \ \ \ \ \ \ \ \ \ \ \ \ \ \ \ \ \ \  &
Expected Output &
P?\\}
\tablehead{\ TC(n)  &
\ Test name \ \ \ \ \ \ \ \ \ \ \ \ \ \ \ \ \ \ \ \ \ \ \ \ \ \ \ \ \ \ \ \  &
\ Test data \ \ \ \ \ \ \ \ \ \ \ \ \ \ \ \ \ \ \ \ \ \ \ \ \ \ \ \ \ \ \ \ \ \ \ \ \ \ \ \  &
Expected Output &
P?\\}
\tabletail{}
\tablelasttail{}
\begin{supertabular}{m{1.186cm}m{7.5950003cm}m{5.727cm}m{3.504cm}m{0.742cm}}
\ 3 \ \ \  &
\ test\_collision\_with\_equal\_mass \ \ \ \ \ \ \ \ \ \ \  &
\ Object(1, 2), Object(1, -3) \ \ \ \ \ \ \ \ \ \ \ \ \ \ \ \ \ \ \ \ \ \  &
(-3, 2) &
~
\\
\ 4 \ \ \  &
\ test\_collision\_with\_different\_mass \ \ \ \ \ \ \  &
\ Object(2, 4), Object(1, -3) \ \ \ \ \ \ \ \ \ \ \ \ \ \ \ \ \ \ \ \ \ \  &
(0.5, 3.5) &
~
\\
\ 5 \ \ \ \  &
\ test\_collision\_with\_one\_stationary\_object  &
\ Object(2, 0), Object(1, -3) \ \ \ \ \ \ \ \ \ \ \ \ \ \ \ \ \ \ \ \ \ \  &
(-3, 0) &
~
\\
\ 6 \ \ \ \  &
\ test\_collision\_with\_opposite\_directions \ \  &
\ Object(1, 2), Object(1, 3) \ \ \ \ \ \ \ \ \ \ \ \ \ \ \ \ \ \ \ \ \ \ \  &
(3, 2) &
~
\\
\ 7 \ \ \ \  &
\ test\_collision\_with\_same\_directions \ \ \ \ \ \  &
\ Object(1, 2), Object(1, 2) \ \ \ \ \ \ \ \ \ \ \ \ \ \ \ \ \ \ \ \ \ \ \  &
(2, 2) &
~
\\
\ 8 \ \ \ \  &
\ test\_collision\_with\_large\_mass\_difference  &
\ Object(100, 5), Object(1, -3) \ \ \ \ \ \ \ \ \ \ \ \ \ \ \ \ \ \ \ \  &
(-3, 5) &
~
\\
\ 9 \ \ \ \  &
\ test\_collision\_with\_zero\_velocity \ \ \ \ \ \ \ \  &
\ Object(2, 0), Object(1, 0) \ \ \ \ \ \ \ \ \ \ \ \ \ \ \ \ \ \ \ \ \ \ \  &
(0, 0) &
~
\\
\end{supertabular}
\end{flushleft}

\bigskip


\bigskip

The evidence is in the appendix :)


\bigskip


\bigskip


\bigskip


\bigskip


\bigskip

\section{Module 3: Drawing some boxes}

\bigskip

Now that we have the underlying architecture in place, we can actually get to drawing boxes, and finish the project.


\bigskip


\bigskip


\bigskip


\bigskip


\bigskip


\bigskip

\section{Appendix}

\bigskip

TC 3-7:


\begin{figure}
\includegraphics[width=17cm,height=9.236cm]{PI-img014.png}\end{figure}
\end{document}
